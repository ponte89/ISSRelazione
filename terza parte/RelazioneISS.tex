\documentclass{llncs}
\let\ifpddf\relax
%%%%%%%%%%%%%%%%%%%%%%%%%%%%%%%%%%%%%%%%%%%%%%%%%%%%%%%%%%%
%% package sillabazione italiana e uso lettere accentate
\usepackage[latin1]{inputenc}
\usepackage[english]{babel}
\usepackage[T1]{fontenc}

%%%%%%%%%%%%%%%%%%%%%%%%%%%%%%%%%%%%%%%%%%%%%%%%%%%%%%%%%%%%%

\usepackage{url}
\usepackage{xspace}

\makeatletter
%%%%%%%%%%%%%%%%%%%%%%%%%%%%%% User specified LaTeX commands.
\usepackage{manifest}

\makeatother


%%%%%%%
 \newif\ifpdf
 \ifx\pdfoutput\undefined
 \pdffalse % we are not running PDFLaTeX
 \else
 \pdfoutput=1 % we are running PDFLaTeX
 \pdftrue
 \fi
%%%%%%%
 \ifpdf
 \usepackage[pdftex]{graphicx}
 \else
 \usepackage{graphicx}
 \fi
%%%%%%%%%%%%%%%
 \ifpdf
 \DeclareGraphicsExtensions{.pdf, .jpg, .tif}
 \else
 \DeclareGraphicsExtensions{.eps, .jpg}
 \fi
%%%%%%%%%%%%%%%

\newcommand{\java}{\textsf{Java}}
\newcommand{\contact}{\emph{Contact}}
\newcommand{\corecl}{\texttt{corecl}}
\newcommand{\medcl}{\texttt{medcl}}
\newcommand{\msgcl}{\texttt{msgcl}}
\newcommand{\android}{\texttt{Android}}
\newcommand{\dsl}{\texttt{DSL}}
\newcommand{\jazz}{\texttt{Jazz}}
\newcommand{\rtc}{\texttt{RTC}}
\newcommand{\ide}{\texttt{Contact-ide}}
\newcommand{\xtext}{\texttt{XText}}
\newcommand{\xpand}{\texttt{Xpand}}
\newcommand{\xtend}{\texttt{Xtend}}
\newcommand{\pojo}{\texttt{POJO}}
\newcommand{\junit}{\texttt{JUnit}}

\newcommand{\action}[1]{\texttt{#1}\xspace}
\newcommand{\code}[1]{{\small{\texttt{#1}}}\xspace}
\newcommand{\codescript}[1]{{\scriptsize{\texttt{#1}}}\xspace}

% Cross-referencing
\newcommand{\labelsec}[1]{\label{sec:#1}}
\newcommand{\xs}[1]{\sectionname~\ref{sec:#1}}
\newcommand{\xsp}[1]{\sectionname~\ref{sec:#1} \onpagename~\pageref{sec:#1}}
\newcommand{\labelssec}[1]{\label{ssec:#1}}
\newcommand{\xss}[1]{\subsectionname~\ref{ssec:#1}}
\newcommand{\xssp}[1]{\subsectionname~\ref{ssec:#1} \onpagename~\pageref{ssec:#1}}
\newcommand{\labelsssec}[1]{\label{sssec:#1}}
\newcommand{\xsss}[1]{\subsectionname~\ref{sssec:#1}}
\newcommand{\xsssp}[1]{\subsectionname~\ref{sssec:#1} \onpagename~\pageref{sssec:#1}}
\newcommand{\labelfig}[1]{\label{fig:#1}}
\newcommand{\xf}[1]{\figurename~\ref{fig:#1}}
\newcommand{\xfp}[1]{\figurename~\ref{fig:#1} \onpagename~\pageref{fig:#1}}
\newcommand{\labeltab}[1]{\label{tab:#1}}
\newcommand{\xt}[1]{\tablename~\ref{tab:#1}}
\newcommand{\xtp}[1]{\tablename~\ref{tab:#1} \onpagename~\pageref{tab:#1}}
% Category Names
\newcommand{\sectionname}{Section}
\newcommand{\subsectionname}{Subsection}
\newcommand{\sectionsname}{Sections}
\newcommand{\subsectionsname}{Subsections}
\newcommand{\secname}{\sectionname}
\newcommand{\ssecname}{\subsectionname}
\newcommand{\secsname}{\sectionsname}
\newcommand{\ssecsname}{\subsectionsname}
\newcommand{\onpagename}{on page}

\newcommand{\xauthA}{NameA StudentA }
\newcommand{\xauthB}{NameB StudentB}
\newcommand{\xauthC}{NameC StudentC}
\newcommand{\xfaculty}{II Faculty of Engineering}
\newcommand{\xunibo}{Alma Mater Studiorum -- University of Bologna}
\newcommand{\xaddrBO}{viale Risorgimento 2}
\newcommand{\xaddrCE}{via Venezia 52}
\newcommand{\xcityBO}{40136 Bologna, Italy}
\newcommand{\xcityCE}{47023 Cesena, Italy}

%
% Comments
%
%%% \newcommand{\todo}[1]{\bf{TODO:}\emph{#1}}


\begin{document}

\title{Ingegneria dei Sistemi Software\\ Approfondimento \\ \footnotesize Terza Parte}

% \author{\xauthA \and \xauthB}
\author{Beatrice Mezzapesa, Alessia Papini, Lorenzo Pontellini}

\institute{%
%%%  \xunibo\\\xaddrCE, \xcityCE\\\email{\{nameA.studentA, nameB.studentB\}@studio.unibo.it}
  \xunibo\\\xaddrCE, \xcityCE\\\email\{beatrice.mezzapesa, alessia.papini, lorenzo.pontellini\}@studio.unibo.it
}

\maketitle

%% \begin{abstract}
%% \footnotesize
%%This a Latex template to be used for the reports of Software Engineering.
%%\keywords{Software engineering, managed software development, reports, ....}
%%\end{abstract}

%%% \sloppy

\section{Problem Analisys and Project}

Riproponiamo alcuni concetti relativi all'analisi del problema e alla definizione e identificazione dell'abstraction gap presentati nella precedente relazione utili contestualizzare questo lavoro:
\begin{quotation}
	\textit{"Visto il contesto nel quale ci si pone, le problematiche identificate hanno portato, come citato all'interno dell'analisi dei requisiti, alla definizione di una serie di concetti per le varie modalit� di esecuzione delle azioni. Occorre identificare una modalit� che ci permetta di astrarre dallo specifico problema in gioco (ovvero quello dei robot) e che ci consenta di definire una soluzione generale alla problematica, sfruttando il robot come approccio, definendo i concetti validi anche per future fasi evolutive del progetto. I requisiti ci portano a riconoscere una serie di problematiche delle quali ci andremo ad occupare: definizioni di azioni sincrone/asicrone ed azioni interrompibili.\\
	Ponendoci nel contesto di esempio, il robot deve essere sensibile ad alcuni cambiamenti che potranno avvenire nell'ambiente circostante nel quale � situato. Occorre, quindi, definire la gestione di questi cambiamenti in modo che durante l'esecuzione delle azioni previste, nel caso di rilevamento di un cambiamento, il robot possa reagire di conseguenza. In particolare si vuole fare in modo che l'azione intrapresa dal robot si blocchi e si possano eseguire azioni alternative, al termine di queste sar� necessario valutare se continuare o meno con l'esecuzione precedente che include le azioni gi� pianificate. Da qui in avanti viene definito \textbf{piano} come sequenza di azioni.\\
	Cos� facendo occorre definire, dato un robot, tutti i cambiamenti ai quali dev'essere in grado di reagire, descrivendo i comportamenti da avere in ogni situazione e le modalit� di controllo che permettano di fare una valutazione sullo stato di avanzamento dell'azione corrente intrapresa dal robot.
	Inoltre � necessario riconoscere quando un'azione � giunta al termine in modo da controllare lo stato dell'azione ed eventualmente proseguire con la successiva.\\
	La base di partenza per noi � l'astrazione di classe \textbf{BaseRobot} il quale tramite apposito metodo ha la possibilit� di eseguire azioni tramite l'uso di appositi attuatori, all'interno del mondo reale. Legata alla problematica identificata, sorge inoltre il problema di specificare la tipologia di azione da utilizzare all'interno del contesto in esame dato che questa, avr� effetti sul controllo del robot stesso e sul comportamento dell'architettura logica creata a valle.\\
	Si vogliono fissare le definizioni relative alle possibili azioni utilizzate all'interno del contesto del problema appena definito:}
	\begin{itemize}
		\item 	\textit{\textbf{Azione sincrona}: si considera un'esecuzione sincrona quando questa viene concretizzata e occorre attendere la terminazione della stessa per poter restituire il controllo al chiamante.}
		
		\item 	\textit{\textbf{Azione asincrona}: si definisce un'esecuzione asincrona quando questa pu� essere eseguita senza che il chiamante debba attendere la fine dell'esecuzione dato che il controllo viene immediatamente restituito al chiamante.}
	\end{itemize}
	
	\textit{Rispetto alla fase precedente si deve considerare un sistema non pi� concentrato ma distribuito, infatti l'utilizzo della console remota rende necessaria la definizione della comunicazione tra quest'ultima e il robot. Dev'essere definito un comando di "halt" che la console sia in grado di inviare al robot e quest'ultimo dovr� essere in grado di interpretarlo."}
\end{quotation}


\section{Future Work}

\newpage

%===========================================================================
\section{Information about the author}
\labelsec{Author}
%===========================================================================

\vskip.5cm
%%% \begin{figure}
\begin{tabular}{ | c | | c || c | }
  % after \\: \hline or \cline{col1-col2} \cline{col3-col4} ...
  \hline
  Alessia Papini & Beatrice Mezzapesa & Lorenzo Pontellini \\
  \hline
  \hline
  
  \includegraphics[scale = 0.6]{img/ale.jpg} &  \includegraphics[scale = 0.6]{img/bea.jpg} & \includegraphics[scale = 0.6]{img/io.jpg} \\
  
  \hline
\end{tabular}


%%% \begin{itemize}
%%% \item Titolo di studio:\\ \\
%%% \item Interessi particolari:\\ \\
%%% \item Ha sostenuto fino ad oggi il seguente numero di esami:\\ \\
%%% \item Deve ancora sostenere i seguenti esami del I anno:\\ \\
%%% \item Prevede di svolgere un tirocinio presso:\\ \\
%%% \item Prevede di laurearsi nella sessione:\\ \\
%%% \item Intende proseguire gli studi per conseguire: \\  \\  \\
%%%   	presso la sede universitaria di: \\ \\
%%% \item Intende entrare subito nel mondo del lavoro presso : \\ \\
%%% \end{itemize}

 
\appendix


\bibliographystyle{abbrv}
\bibliography{biblio}

\end{document}